\documentclass[10pt]{article}
\author{by: Kyle Lemons}
\title{A Proposal for Generic Types in Go}

\usepackage{color}
\usepackage{listings}
\usepackage{upquote}
\usepackage{caption}
\usepackage{fontenc}
\usepackage{courier}
\usepackage{hyperref}

% Set up cross references and links
\hypersetup{
  hyperindex,
  colorlinks,
  allcolors=blue,
  urlcolor=red,
  filecolor=red,
  bookmarks,
  bookmarksopen=true,
  bookmarksnumbered=false,
}

% Define the Go language for listings
\lstdefinelanguage{Go}{
  sensitive=true,          % Keywords are case sensitive
  keywords={               % Language keywords
    break,        default,      func,         interface,    select,
    case,         defer,        go,           map,          struct,
    chan,         else,         goto,         package,      switch,
    const,        fallthrough,  if,           range,        type,
    continue,     for,          import,       return,       var,
    generic,
  },
  morekeywords=[2]{
    uint, uint8, uint16, uint32, uint64, byte, uintptr,
    int, int8,  int16,  int32,  int64,
    float32, float64, complex64, complex128,
  },
  comment=[l]{//},         % Line comments
  morecomment=[s]{/*}{*/}, % Block comments
  string=[b]",             % Normal strings, allow [b]ackslash escapes
  morestring=[b]',         % Character literals, allow [b]ackslash escapes
  morestring=[s]``,        % Raw strings, don't allow any escapes
  literate=*{::}{$\cdot$}1,  % Allow :: instead of the cdot
}

% Define some pretty colors
\definecolor{keyword}{RGB}{154,61,61} % Dark red
\definecolor{type}{RGB}{154,119,61}   % {255,128,64} % Dark Orange
\definecolor{string}{RGB}{45,69,101}  % Dark blue
\definecolor{comment}{RGB}{49,123,49} % Dark green

% Define a pretty style for code
\lstset{
  language=Go,                % Select the language to use
  % Styles
  basicstyle=\small\ttfamily,    % Use small monospace text
  stringstyle=\color{string},    % Strings are blue
  keywordstyle=\color{keyword},  % Keywords are red
  keywordstyle=[2]\color{type},  % Types are orange
  commentstyle=\color{comment},  % Comments are green
  showspaces=false,              % Don't show spaces as underbrace
  showstringspaces=false,        % Don't show spaces in strings
  % Line numbers
  numbers=left,               % Number lines on the left
  numberstyle=\tiny,          % Make the line numbers small
  stepnumber=1,               % Number every line
  numbersep=5pt,              % Put space between the number and the code
  % Tabs and alignment
  tabsize=2,                  % Default tab width
  showtabs=false,             % Don't show tabs as arrows
  breaklines=true,            % Add line breaks for long lines
  breakindent=12px,           % Indent broken lines
  breakatwhitespace=true,     % Try to break lines at whitespace
  % Misc
  extendedchars=true,         % Allow UTF-8 characters
  % Framing
  frame=b,                    % Show a bottom frame for listings
  xleftmargin=17pt,           % Add some extra space for the code
  framexleftmargin=17pt,      % Widen the frame to accomodate
  framexrightmargin=6pt,      % Widen the right side too
  framexbottommargin=3pt,     % Leave some whitespace below the text
  % Use math escapes
  mathescape,
}

% Make the pretty listing blocks
\DeclareCaptionFont{white}{\color{white}}
\DeclareCaptionFormat{listing}{\colorbox[cmyk]{0.43, 0.35, 0.35,0.01}{\parbox{\textwidth}{\hspace{15pt}#1#2#3}}}
\captionsetup[lstlisting]{
	format=listing,
	labelfont=white,
	textfont=white,
	singlelinecheck=false,
	margin=0pt,
	font={bf,footnotesize}
}

% Commands
\newcommand{\code}[1]{\texttt{#1}}
\newcommand{\Listing}[1]{\autoref{lst:#1}}

\begin{document}

\maketitle
\tableofcontents
\lstlistoflistings
\newpage
\section{Introduction}
I have been using Go now for over a year.  I think that through this time, I have gotten a pretty good feel for the language as a whole and the principles that guide it.  There have been very few times where I have even been tempted to look for generic types while writing actual programs, but for some things (like the math package), there is clearly no good solution yet within the confines of the language.  The creators of Go have acknowledged that they will consider adding generic types to the language if they find the right way to go about it, so here is my humble attempt at defining what generic types could look like within Go.

\subsection{Philosophy}
In designing this proposal for generic types, there are a few fundamental principles which have guided my hand.  I have discussed various generic type proposals with other members of the community who have other design goals in mind, so I think it's probably a good idea to state these up front.
\begin{itemize}
\item Generic types should be simple.
\item Generic types should not add any new syntax to the language.
\item Generic types should not incur much, if any, run-time overhead.
\item It should be possible to use a generic value as a map key or slice index.
\item Generic types should not require the programmer to explicitly parameterize the Generic types (as templates in C++).
\item Generic types should not require importing a package multiple times for multiple, different instances of the generic construct in question.
\end{itemize}

\subsection{Overview}
This proposal for generic types adds a single keyword to the language: \code{generic}.  The keyword can be used in a type definition, in place of an interface or structure type.

Consider a fictitious package (say \code{gen}) in which the code from \Listing{basic} is used.
\begin{lstlisting}[float=tbp,caption={Basic example of generic types},label=lst:basic]
// Package gen provides some example generic functionality.
package gen

type Number generic

// Min returns the number closest to -$\color{comment}\infty$.
func Min(a, b Number) Number {
	if a < b {
	  return a
	}
	return b
}

type Value generic

// Tern is analogous to the C ternary value syntax:
//   cond?t:f
func Tern(cond bool, t, f Value) Value {
  if cond {
    return t
  }
  return f
}

// Check can be used to do simple error checking of
// function return values.
func Check(ret Value, err os.Error) Value {
  if err != nil {
    panic(err)
  }
  return ret
}
\end{lstlisting}

\end{document}